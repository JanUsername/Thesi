\chapter{Implementation}
\section{OTA / tdconnect, Visual Basic Script}
Die Open Test Architecture API (OTA) erlaubt es mit tdconnect Zugriff auf HP
Application Lifecycle Manager �ber externe Scripte zu haben. Die Schnittstelle
wird mit Visual Basic Script angesprochen. F�r diesen Zweck wurde Visual Basic
Script C(sharp) bevorzugt da diese Sprache f�r HPQC am geeignetsten ist. VBS ist
gut dokumentiert und verf�gt �ber eine viel gr��ere Bibliothek was die Steuerung des
Tools anbelangt. Der Workflow von HPQC ist auch mit VBS beschrieben, was eine
Andockung mit der selbigen vereinfacht. C(sharp) scheint zu diesen Zeitpunkt
noch nicht f�r die Software ausgearbeitet genug zu sein. 

Creating a connection
\lstset{language=VBScript}

\begin{lstlisting}
' Create a Connection object to connect to Quality Center
   Set tdConnection = CreateObject("TDApiOle80.TDConnection")
'Initialise the Quality center connection
   tdConnection.InitConnectionEx qcURL
'Authenticating with username and password
   tdConnection.Login qcID, qcPWD
'connecting to the domain and project
	tdConnection.Connect qcDomain, qcProject
\end{lstlisting}

�ber dieses tdConnection Objekt k�nnen wir Objekte f�r HP ALM erstellen,
l�schen oder ver�ndern. In unseren Fall sind das selbstgebaute Listen.

\section{ Listen adden + Elemente}
\begin{lstlisting}
' custom is the connection with the elements, oList are the lists access
   Dim custom
    Set custom = tdConnection.Customization
    Dim oLists
    Set oLists = custom.Lists

    'add list for each
    for each item in listToAdd 
                               oLists.RemoveList(lhtListS & item)
                               oLists.AddList(lhtListS & item)
                               'AddList RemoveList; Adding or removing List to the queue on the list object
                next
    custom.Commit
End Sub
\end{lstlisting}


In diesen Beispiel kann man erkennen das die Listen bevor sie geuploadet
werd erst gel�scht werden. Was erst Gegenintuitive erscheinen mag wird so
gehandhabt weil Listen die es bereits gibt einfach �bersprungen werden. Das hat
zur Folge, falls eine Liste nuie Elemente hinzugef�gt bekommt, w�rde diese nicht
aktualisiert werden. Praktisch hier kommt das HP-Alm beim L�schen von Listen den
Befehl einfach �berspringt falls die Liste nicht vorhanden ist. So kommt es zu
keinen Problemen.

\section{Aus XML auslesen}
\begin{lstlisting}
for each child in xmlDoc.SelectNodes("/qtpRep:ObjectRepository/qtpRep:Objects/qtpRep:Object")
  objectLevelName = child.getAttribute("Name")
  objectLevel1.add(filenameXML & "|" & objectLevelName)
  
  'first iteration, getting the second level
  For Each child2 In child.SelectNodes("./qtpRep:ChildObjects/qtpRep:Object")
    objectLevel2Name = child2.getAttribute("Name")
    objectLevel2.add(objectLevelName & "|" & objectLevel2Name )
    
    'third iteration, getting the objects
      For Each child3 In
      child2.SelectNodes("./qtpRep:ChildObjects/qtpRep:Object") objectName =
      child3.getAttribute("Name") objectList.add(objectLevel2Name & "|" & objectName )
      
      'fourth iteration, getting the objects in case objects have objects
        For Each child4 In
          child3.SelectNodes("./qtpRep:ChildObjects/qtpRep:Object") objectName2
         = child4.getAttribute("Name") objectList2.add(objectName & "|" & objectName2 ) 
       next
     next
   next
next
\end{lstlisting}

Das Herzst�ck dieses Projektes, das Durchlaufen der XML. Der erste Schritt ist
das auslesen aller Knotenpunkte mit dem Befehl Select Nodes. Dieser beinhaltet
zuert allte Objekte in einer einzigen Dimension. Wir k�nnen jetzt den ersten
Punkt in die Liste als root Listenname setzen. Dieses Rootelement besteht aus
dem Dateinamen und dem Attributname. Hier gilt zu beachten das maximal 4
Tiefen existieren wobei nur 3 �blich sind und die letze ein Ausnahmefall ist.
Dieser kann bei einer Combobox mit vorgefertigten Elementen vorkommen.

In diesen Fall haben wir eine rekursive linked List erstellt ohne abstrakte
Datentypen, sondern nur �ber den Namen, da VBS diese nicht native unterst�tzt. 
An der linken Seite der Pipe sehen wir wo wir gerade sind, auf der rechten das
gew�nschte Objekt. Der Listenname gibt an an welchen Panel/Browser wir uns
gerade befinden. 
Diese Manipulation ist n�tig da beim einlesen die Level der XML verloren gehen
und alles Eindimensional dargestellt wird und wir nur Knotenpunkt f�r
Knotenpunkt linear verarbeiten k�nnen. 

Die Variante wurde einem rekursiven Aufruf bevorzugt da die Schreibweise
intuitiver erscheint und damit leichter zu warten ist. 

\section{Listnamen Manipulation}
\begin{lstlisting}
for each list in listToAdd
listName = getRightSideOfPipe(list)
for each item in itemToAdd 
  objectList = getLeftSideOfPipe(item)
  if(listName = objectList) Then
    if oLists.IsListExist(lhtListS & list) Then
    Set myList  = oLists.List(CStr(lhtListS & list))'nameList
    Set listRoot = myList.RootNode 
    objectName = getRightSideOfPipe(item)
    On Error Resume Next
      listRoot.AddChild(CStr(objectName))'subname
    On Error GoTo 0
   
    else 
      MsgBox("The List: " + lhtListS & list + " does not exist or is not yet commited") End If
    End If
    next
  next
custom.Commit
\end{lstlisting}

Der Namen ist mit einer Pipe ``|'' getrennt. Auf der rechten Seite zeigt dieser
an welches Objekt gerade im Focus steht. Im Gegensatz auf der linken Seite steht
das Objektlevel, also im welchen Panel, Browser etc das Object platziert ist.
Dieser funktioniert auch als Listenname. Sobald der Listenname nicht mehr mit
dem Objectlevel �bereinstimmt, kreieren wir eine neue Liste.

Der Commit ist der Teil des Programms der am l�ngstem braucht, viele Commits
w�rden viel Zeit beanspruchen, deshalb wird dieser erst ausgef�hrt sobald wir
alle Ebenen und Objekte im Zwischenbuffer oList abgelegt haben.

 *instert graphic with name description

\section{Ergebnis}