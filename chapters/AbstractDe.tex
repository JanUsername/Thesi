\chapter*{Abstract deutsch}

An abstract should be structured as
follows\footnote{https://www.ece.cmu.edu/~koopman/essays/abstract.html}:

\begin{enumerate}
  \item 
 \item \textbf{Problem statement}: Den Verlauf der Testautomatisierung kann man in 2 Phasen unterteilen. Erstere
 die Dokumentation der Testf�lle und letztere das Automatisieren. W�hrend
 die bereits standartisierten Tests aus der Testfactory gewartet werden riskiert
 man durch den st�ndigen Wandel der zu Testenden Programme die Dokumentation zu
 vernachl�ssigen da diese nicht nur Zeitintensiv ist sondern auch nicht so oft
 ben�tigt wird wie ersteres, aber nicht weniger wichtig.
  

  \item \textbf{Motivation}: Die Motivation hinter diesem Projekt ist die Wartbarkeit der Dokumentation zu
  erleichtern bzw. zu erm�glichen. Dessen Funktion vielleicht nicht so viel
  Einfluss auf den Alltag hat, jedoch gro�e Hilfe bieten kann f�r den Fall das
  Fehler entdeckt werden und/oder der zust�ndige f�r den Bereich nicht Anwesend
  ist.
  

  \item \textbf{Approach}:
  Die ersten Schritte bei der Probleml�sung war das Erfassen was bereits
  vorhanden ist womit man Arbeiten kann und was selbst noch zu machen ist.
  Sobald man ein Gef�hl daf�r hat macht man sich mit der Umgebung vertraut, wie
  kann man was Steuern, welche Tools k�nnen helfen, welche Sprache ist hier
  am besten geeignet. Die Wahl fiel schnell auf VBS das es von der Applikation
  am besten unterst�tz wird. Obwohl diese in viele Hinsicht C-Sprachen
  unterlegen w�re.
  Hier ist auch n�tzlich zu Wissen wie weit man das System auslasten kann. Da
  wir hier von tausenden Elementen sprechen m�ssen wir schaugen ab wann wird
  der Ablauf durch die neue Feature langsamer oder es Limitationen vom System aus gibt.
  
  Hier hatten wir noch dazu die M�glichkeit das Tool intern oder extern zu
  erweitern, und entschieden uns auf so viel M�glich auf extern, da sich intern
  bereits �ber die Zeit sich viel Code angesammelt und die externen VBS Scripte
  einfach einzuspielen sind.
  
  Bei der Implementation ist aufzupassen welche Fehler passieren k�nnten und wie
  man mit ihnen Umgehen sollte. Da dieses Projekt gro�es Erweiterungpotential
  hat war es auch wichtig versuchen im Rahmen der Zeit zu bleiben.
  
  

  \item \textbf{Results}:   
  Das Erebnis ist eine leichtere und weniger Fehleranf�llige Eingabe bei der
  Dokumentation. Auch hat man sp�ter eine leichtere Wartung da der Pfad
  standartiesiert eingeben ist und man so Testf�lle ausmachen kann die von der
  �nderung bestimmter Masken betroffen sind. �ltere Systeme m�ssen �berarbeitet
  werden da die Daten in verschiedenen Formaten abgespeichert worden sind.

  \item \textbf{Conclusions}: What are the implications of your answer? Is it
  going to change the world (unlikely), be a significant "win", be a nice hack,
  or simply serve as a road sign indicating that this path is a waste of time
  (all of the previous results are useful). Are your results general,
  potentially generalizable, or specific to a particular case?
  
  Das erreichte Ergebnis bietet die Grundlage einer besserung Wartung. Die
  erleichtert die Eingabe bietet aber sogleich M�glichkeit ausgebaut zu werden
  um mehr Arbeiten weniger Fehleranf�llig sein zu lassen oder gar zu
  automatisieren.
  
\end{enumerate}