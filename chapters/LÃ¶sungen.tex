\chapter*{L�sungen}

4. L�sung
4.1 Object Repository
Im Fokus steht ein Object Repository f�r Object Path und Object Name. Pro Eintrag in dieser Liste muss jedes Element genau identifiziert sein. Aus diesem soll der User sp�ter die Eintr�ge vom Element in das HPQC nehmen. Zugriff hat dieser direkt im HPQC bei der Eingabe-Maske �ber ein Drop-Down Men�.
4.2 UTF XML
Die Liste wird dynamisch aus bereits automatisierten Tests erstellt. Zugriff darauf hat man mit UFT (Unified Functional Testing). �ber diese Software kann man �ber bereits erstellte Scripts in VBS die angesprochenen Elemente auslesen. Hier kann man alle Test-Scripts ansprechen und hat somit alle Elemente die angesprochen werden abgedeckt. Die Ausgabe erfolgt in einem XML Format.
4.3. Workflow
Der Workflow ist eine Schnittstelle an der man sich anbinden kann um die �bertragung ins HPQC zu ver�ndern. In unseren Fall die Eingabemaske. Einfache Textfelder (in HTML Editfields) werden durch Listen ersetzt. Aus diesen w�hlt man sich das Ergebnis gew�nschte Ergebnis. So ist es m�glich ein Element zu ver�ndern und �ber alle Testf�lle �bergreifenden �ndern mit einem einzelnen Eingriff in das Object Repository. (wo denn?). Parallel dazu wird die Fehlerquote reduziert, die mit manueller Eingabe vorkommen.
4.4. Binary Repr�sentation
Der Object Path wird in Zukunft mit mehreren Feldern angezeigt, dass die Eingabe erleichtert. Wird eine Applikation ausgew�hlt gibt es im ersten Object Path Feld nur die Elemente auszuw�hlen die auch in der Maske sind. So geht es mit Unterseiten oder Reitern weiter bis wir beim gew�nschten
Objekt sind. Dieser wird mit der Element- Art verkn�pft, sprich Button, Editfield usw. Das soll dazu f�hren das nur die Objekte dargestellt werden die auch ausgew�hlt werden die auch mit der vorher eingegeben Objekt Art �bereinstimmen. Vor dieser Auswahl bleibt das Feld Object Name ausgegraut. Es entsteht eine Baumstruktur die das Navigieren in diesen einfacher gestaltet als eine gro�e Liste.