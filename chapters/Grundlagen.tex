\chapter{Grundlagen und Stand der Technik}
\section{Hp Tool}
Wenn du ein �HP Tool� erw�hnst, musst du dieses Tool kurz erkl�ren. Das w�rde in �Grundlagen und Stand der Technik� gut hineinpassen.
Das HP Tool Application Life Cycle Manager oder kurz HP ALM ist eine Sammlung von Software Applikationen die den Lebenszyklus von Software unterst�tzt.
 In diesen Fall die Erstellung von automatisierten Testf�llen. HP. 
 Spezifisch wird hier HP Quality Center verwendet das eine einzige Plattform f�r
 Test- und Software Zyklus Management bereitstellt. Die Aufgabe der Software
 Factory ist nicht die genutze Software nachzubasteln sondern eigentst�ndige
 Programme schreiben die auf einer existierend Testumgebung die gew�nschten
 Abl�ufe rekonstruiert. 



 
 
 
 TODO:
 -> Outsourcing theorie, zumindest a bissl
  4) Grundlagen -> Formale Sprache, Object Path ersatz, find that document->gefindet

  [insert source and more explanation about hp here + Cycles?]
\section{UFT}