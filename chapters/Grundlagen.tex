\chapter{Grundlagen und Stand der Technik}
\section{Hp Tool}
Das HP Tool Application Life Cycle Manager, oder kurz HP ALM, ist eine Sammlung
von Software Applikationen die den Lebenszyklus einer Software unterst�tzt.
 In diesen Fall die Erstellung von automatisierten Testf�llen.  
 Spezifisch wird hier HP Quality Center verwendet das eine einzige Plattform f�r
 Test- und Software Zyklus Management bereitstellt. Aufgabe der Software
 Factory ist nicht die genutze Software nach zu bauen sondern eigentst�ndige
 Programme schreiben die auf einer existierend Testumgebung die gew�nschten
 Abl�ufe rekonstruieren. 

\section{Daten}
Eine Sammlung von allen Objekten die bei der Automatisierung angesprochen worden
sind gibt es in Form von XML-Dateien.  Diese liegen extern und nicht im HP-QC
selbst. Da sie von der Testfactory angefertigt worden sind, eignen sie sich
optimal da selbige beim Automatisieren verwendet und somit 
Konflikte mit der Namensgebung der Elemente vermieden werden.

 Da reine XML nicht in dieser Form eingespeichert und verwendet werden k�nnen
 m�ssen diese erst geparst werden. Am besten eigenen sich hier ``Project List''
die von HP bereitgestellt werden. Diese k�nnen auch mit externen Scirpten
erstellt,gel�scht oder bearbeitet werden. Intern sind diese einfach aufrufbar
und somit ideal geeignet um das Eingabefenster je nach Aufruf dynamisch zu
ver�ndern.
\begin{figure}[h]
\includegraphics[width=\textwidth]{\string"img/project lists".png}
\caption{ In diesem Bild sehen wir ein Beispiel von HP eigenen Listen. Diese
k�nnen vom Workflow aus zugegriffen werden und von extern bearbeitet oder
erstellt werden.}
\end{figure}

Der Workflow reguliert den Ablauf der �bertragung der Testf�lle ins HPQC. Mit
diesen kann man Felder an Bedingungen kn�pfen, lehr lassen, ausgrauen,
autovervollst�ndigen etc. 

\section{HP ALM XML}
\begin{lstlisting} [caption=Auschnitt aus einer XML Datei,label=XML]
listen,label=lst:listing connect] <qtpRep:ObjectRepository xmlns:qtpRep="http://www.mercury.com/qtp/ObjectRepository"> <qtpRep:Objects>
<qtpRep:Object Class="Link" Name="matRequest_info_second_IBO_Materialstamm_Tab">
<qtpRep:ChildObjects>
<qtpRep:Object Class="WinButton" Name="matRequest_info_NotificationBar_FileSave_OpenButton">
\end{lstlisting}

Im oberen Beispiel k�nnen wir die Grundstruktur der XML Dateien erkennen. In der
ersten Zeile wird die Repository URL angegeben. F�r unsere Listen ben�tigen wir
den Unterpunkt ``qtpRep:Objects''. In diesem werden alle Objekte,  von Buttons
�ber Explorer eingeordnet. Die sind unter ``qtpRep:Object'' zu erkennen.

 Die wichtigsten Elemente die wir aus der XML auslesen k�nnen sind die
 Attribute Class und Name. Das Attribut Class gibt mit welche Art von Objekt wir es hier zu
tun haben. Im Code-Beispiel ist es einmal ein Link und darunter ein
Button.
Das n�chste ist f�r uns das wichtigste Element, der Namen mit der ein
Objekte eindeutig gekennzeichnet wird. Am Ende von diesen steht auch eine Art
Klasse in der genau spezifiziert wird was dies darstellt.

Anklickbaren Elemente wie Textboxen etc. werden alle erst untergeordnet in
Explorer, Applikation etc.  unter Childobjects. Ein gr��erer Panel besteht
immer aus mehreren kleinenren Interaktiven Elementen. Diese k�nnen selbst Unterelemente haben.  Mit der
Klasse, Namen und dem Pfad werden sie eindeutig verifiziert und zugeordnet.
werden.

\section{Methodisches Vorgehen}
F�r dieses Projekt benutzen wir die f�r diese Art von Aufgabe untypische
Programmiersprache Visual Basic Script. Aus dem einfachen Grund, dass man mit
dieser einfach auf den Lifecycle Manager zugreifen kann. Auch C(sharp)
bietet diese M�glichkeit ist aber im Funktionsumfang und Ausgereiftheit deutlich
 VBS unterlegen. Auch sind VBS Scripte im HP Alm einfacher einzubinden. 

Noch dazu bietet HP ALM eine einfache Schnittstelle f�r VBS um mit dem Programm
auch von extern kommunizieren zu k�nnen, die Open Test Arcitecture. Der Vorteil
von extern auf die Software zuzugreifen liegt auf dementsprechend externe
Dokumente leichter Zugriff zu haben. In unseren Fall XML-Files. Diese Prozedur
wird unter Implementation genauer erkl�rt.

Ein weitere Vorteil die Scripte nicht direkt in den Hauseigennen ``Worflow''
einzubinden ist das dieser schon mehrere Tausend Zeilen schwer ist. Dies w�rde
die �bersicht des bereits bestehenden Code verringen.

Bevor wir mit der Implementation besch�ftigen, m�ssen wir und vorerst die
Einschr�nkungen unseres Programmes kennen. Unter diesen versteht man
hier die maximale Azahl an Elemente die das Programm erfassen kann, ebenso
die in einer einzigen Iteration uploaden kann oder ob die Ladezeiten dabei
nicht zu gro� werden. Auch d�rfen diese sp�ter nicht die Performance f�r User
beintr�chtigen.

Die Listen die eingetragen werden m�ssen werden auf unter >10.000 gesch�tzt. Die
Tests mit HP ALM zeigen keine negativen Auswirkungen auf HP ALM auch �ber
100.000 Elementen. Auch die Einspielzeit, sofern es in wenigen Iteration
geschieht, bleiben immer unter 15 Minuten. Um bei Listen Elemente hinzuzuf�gen,
wie ein Button etc. muss diese aber bereits im HP QC commited sein, nicht nur
hinzugef�gt. Da aber Listen einzeln commiten mehrere Stunden dauern w�rde muss
man die einzelnen Elemente sp�ter hinzuf�gen.

