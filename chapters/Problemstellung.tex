\chapter{Problemstelliung}

\section{Warum HP Quality Center}
Durch eine einzige Plattform die dieses Tool f�r Lebenszyklen und Test
Management bereitstellt macht es f�r die Testautomatisierung geeignet. Die
Testdokumentation und dessen Automatisierung folgt gewissen Zyklen [explaining
here or somewhere else, or at all? ] die sich in HP QC unter �Status�
wiedergeben l�sst. Das Projekt ist nicht zentral und wird teilweise outsourcend
und bei Setzung dieses Status wei� wann welche Personen jetzt f�r den n�chsten
Schritt verantwortlich sind. Diese werden benachrichtigt sobald dieser in den
eigenen Aufgabenbereich f�llt.


\section{Aufgabenstellung}
Die Problemstellung ist eine Feature zu erstellen mit dem wir die Eingabe der
ELemente in den Anwendungen erleichtern. Dieser muss �ber die Eingabemaske in
HPQC erfolgen. Die Daten sollen aus der XML gelesen und im HP QC
eingespielt werden. Das Erfordert ein Parsen der XML Dateien. Dieses dann
erstellte Object Repository soll dann �ber die Eingabemaske aufrufbar sein.


\section{Requierments}
Im Rahmen dieser Arbeit wird �ber UFT eine XML aus den bereits automatisierten
Testf�llen alle bekannten Objekten ausgelesen und �ber in ein Object Repostory
eingespielt. Auf diesen wird �ber den ALM Explorer mit Dropdown Men�s dem
Endnutzer Zugriff gegeben. Der gesamte Pfad zu einem Objekt hin muss �ber
dieselbe HP QC GUI erkenntlich sein.
Es muss eine M�glichkeit bestehen Objekte die neu hinzugekommen sind trotzdem
einzuspielen. Am besten ohne externe Anbindungen und direkt �ber den Input an
der grafischen Oberfl�che.
Im Besonderen darf das Projekt HP QC nicht deutlich langsamer machen. Wird das
Tool bis zur Unbrauchbarkeit langsamer �berwiegen die Nachteile den Vorteilen
und das Projekt wird �berfl�ssig.


\section{Requirements}
\label{sec:Requirements}

If you are building a system, define here the concrete requirements to build it.
Above you described which problem you want to address, here you write
\textit{how} you designed the system to address it.

Follow the approach of \cite{Lauesen2002}, begin with the goal requirements:

\begin{enumerate}[label=\bfseries G\arabic*:]
  \item Das Ziel dieser Erweiterung ist eine Erh�hung in Qualit�t f�r die
  Dokumentaion von Tests der Anwendungunen f�r die Lufthansa Supply Chain.
\end{enumerate}

Then define domain requirements:

\begin{enumerate}[label=\bfseries D\arabic*:]
  \item Der User soll den Pfad einzelner ELemente �ber Dropdown Men�s ausw�hlen
  k�nnen
  \item Es sollen vier Level an Dropdown Men� geben, ausschlie�end den Namen und
  f+r welche der Applikation Tests geschrieben werden sollen
  \item Reg222
\end{enumerate}

Product requirements:

\begin{enumerate}[label=\bfseries P\arabic*:]
  \item Die XML Dateien m�ssen in ein Format gebracht werden mit dem HP QC
  arbeiten kann
  \item Der Output muss Stufenweise passend zum XML Format
\end{enumerate}

Design requirements (or constraints):

\begin{enumerate}[label=\bfseries C\arabic*:]
  \item Die XML Dateien k�nnen nicht ge�ndert werden, das Format muss
  beibehalten werden
  \item HPQC darf beim einspielen nicht merklich langsamer werden
  \item Der End User soll keine Wartezeiten beim Laden der Felder f�r den Path
  habe 
  \item Die Programmiersprache soll VBS
  \item Das Sytem soll Fehler erkennen und so fr�h wie m�glich diese Melden
  \item Die Erweiterung soll ausbaubar sein
  \item Der im Workflow geschriebene Code muss kombatibel mit dem restlichen da
  angereichtern Code sein, ohne etwas vom Verhalten zu ver�ndern
  \item Es muss auf dem Betriebssystem Windows  10 funktionieren 
  \item Elemente m�ssen eindeutig Verifizierbar sein
\end{enumerate}

When defining requirements, you can define functional requirements and also
quality requirements. The quality requirements are then refined using quality
scenarios in the next section.

\subsection{Quality scenario}
