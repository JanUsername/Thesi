\chapter{Problemstelliung}

\section{Problemstellung}
\section{Warum HP Quality Center}
Durch eine einzige Plattform die dieses Tool f�r Lebenszyklen und Test
Management bereitstellt macht es f�r die Testautomatisierung geeignet. Die
Testdokumentation und dessen Automatisierung folgt gewissen Zyklen [explaining
here or somewhere else, or at all? ] die sich in HP QC unter �Status�
wiedergeben l�sst. Das Projekt ist nicht zentral und wird teilweise outsourcend
und bei Setzung dieses Status wei� wann welche Personen jetzt f�r den n�chsten
Schritt verantwortlich sind. Diese werden benachrichtigt sobald dieser in den
eigenen Aufgabenbereich f�llt.
\section{Requierments}
Im Rahmen dieser Arbeit wird �ber UFT eine XML aus den bereits automatisierten
Testf�llen alle bekannten Objekten ausgelesen und �ber in ein Object Repostory
eingespielt. Auf diesen wird �ber den ALM Explorer mit Dropdown Men�s dem
Endnutzer Zugriff gegeben. Der gesamte Pfad zu einem Objekt hin muss �ber
dieselbe HP QC GUI erkenntlich sein.
Es muss eine M�glichkeit bestehen Objekte die neu hinzugekommen sind trotzdem
einzuspielen. Am besten ohne externe Anbindungen und direkt �ber den Input an
der grafischen Oberfl�che.
Im Besonderen darf das Projekt HP QC nicht deutlich langsamer machen. Wird das
Tool bis zur Unbrauchbarkeit langsamer �berwiegen die Nachteile den Vorteilen
und das Projekt wird �berfl�ssig.
[Noch nicht sicher wo das hingeh�rt]