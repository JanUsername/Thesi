\chapter*{Abstract}

\begin{enumerate}
  \item 
 \item \textbf{Problem statement}: 
 You can divide the lifecycle of a test automatation into two Phases. First the
 documentation of a test and at last the actually automatation. While the
 already standartised and running test get mantainened in the test factory the
 documentation could be neglected. This could be because a even a small change
 in the interface of an Application which is test created can take a lot of work
 to correct every path of affected testcases.	
  

  \item \textbf{Motivation}:   
  The motivation behind this project lays in increasing the maintainability of
  the documentation. Which function mein non have an inpact on the routine of
  daily tests but are of help when they fail or when the expert on this field is
  not available.

  \item \textbf{Approach}:
  The first step in this approach was to understand what is already in place
  which could prove useful and what has to be made from scratch. Once familiar
  with the environment you have to find what which elements can be
  used with which interface. Which tools could saef time and which programming
  language proves the most efficient. In this case it was VBS since HP made
  Interfaces for it, even it is objectly not as flexible as for examples C
  languages.
  Important to know how much the system can handle before slowing down
  noticeable or if they have any restriction on how much can be saved inside it.
  This is all important since we are going to use more than 10.000+ Elements.
  
  A decision had to be made pretty soon into the project. Either using extern
  tools to manage processes inside the HP Application or from the intern
  workflow wich is an interface made from HP to insert companies own code an
  allowing customizations to happen. It was decided against intern since a lot
  of code has already accumulated there, therefore risking to causing cunfusion.
  
  While implementing one had to consider which kind of errors could happen
  during runtime and how the system should handle it. One last importantn factor
  to consider was timemanagment. This extension has potential to becoming
  bigger and providing more help to the enduser or even starting automated
  processes on it's own. So it's was important setting feasable goals.
  

  \item \textbf{Results}:   
  Das Erebnis ist eine leichtere und weniger Fehleranf�llige Eingabe bei der
  Dokumentation. Auch hat man sp�ter eine leichtere Wartung da der Pfad
  standartiesiert eingeben ist und man so Testf�lle ausmachen kann die von der
  �nderung bestimmter Masken betroffen sind. �ltere Systeme m�ssen �berarbeitet
  werden da die Daten in verschiedenen Formaten abgespeichert worden sind.
  
  The result is a simplification for the input an enduser makes writing down the
  path where a certain element lays. The maintainance work of keeping up with
  changes are made less wasteful since a standartised way to keeping the path
  makes it easier.

  \item \textbf{Conclusions}:    As conclusion the extensions offers a easier maintainance which results in a
  higher quality of it. It can act as an base for further improvement too. That
  would go to starting already on it's own the automating process.
  
\end{enumerate}