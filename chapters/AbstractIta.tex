\chapter*{Abstract}


\begin{enumerate}
  \item 
 \item \textbf{Problem statement}: 
 Il percorso che prende un intera automatisatione di un Test si pu� dividere in
 due fasi. Per primo la documentatione dei test e poi la automatizzazione.
 Mentre i test gia automatizzati vengono mantenuti dai automatizzatori esterni
 la documentatione che compresi i test manuali rischiano di non essere mantenuti
 a dovere. Un motivo puo � che i cambiamenti alle interfaccio comprendono un
 lavoro molto esteso per aggiornorli la documentatione.
  

  \item \textbf{Motivation}: 
  Lo scopo dietro questo progetto � semplificare la manuntenzione della
  documentatione. Questa non ha molta influenza nei test giornaleri ma � un
  grande auito in caso die errori di essi, o se l�esperto non � disponibile in
  un preciso momento.
  

  \item \textbf{Approach}:
  I primi passi nella soluzione � capire cosa si possa usare e cosa esiste gia.
  Cosi si evita di fare lavori doppi. Il prossimo � comprendere come funzione
  l'ambiente in cui si lavoro. Come i elemtno si riescono usare, quali strumenti
  possono essere di auito, quale lingua di programmazione � la piu efficente. In
  questo caso �ra VBS visto che � quella prevista da HP anche con tutte le
  disvantaggi.
  Importante � sapere quanto il systema pu� essere usato senza rallentarlo o
  se ci fossere restrizioni in termini di memoria visto che usiamo quantita di
  dati piutosta elevata.
  
  Una decisione da prendere subito dall'inizio � se programmare zu fonti esterni
  o interni all programma. Nel nostro caso preferiamo programmare all'esterno
  del Quality Center visto che nel corso del tempo si � accumulato molto codice
  all'interno � rischeremo di fare confusione.
  
  In un ultimo passo � importante fare attenzione ad eventuali errori e come
  reagire su quali. Questa estenzione ha grandi potentialit� di diventare pu�
  grande e pi� utile in future, per questo � importante quanto si riesce a fare
  nel periodo di tempo dato.
  

  \item \textbf{Results}:   

  Il risultato di questo � una input di dati facilizata per il user comune,
  diminuendo anche i errori. La manuntezione comprende meno lavoro visto di
  trovare i test convolti di una cambiazione con dati standartizzati � meno
  difficile.

  \item \textbf{Conclusions}: 
  Per concludere l'estenzione fornisce una buona base per futuri aggiornamenti
  comprendendo una manuntenzione di migliore qualita o possibilimete piu lavori
  automatizzati.
  
\end{enumerate}