\chapter{Herausforderungen}

[In der Zusammenfassun so lange nicht aktive Probleme. Dargestellt wie in Zukunft ausbauf�hig und k�nnte und den Umst�nden Probleme machen. Alles komplett abschlie�en, keine Offenen Enden]
\section{Rechte in einen Gro�konzern} 
W�hrend der Entwicklung des tools wird das Hp Quality Center Programm noch
weiterhin verwendet. In diesen sind Daten gespeichert die nicht ver�ndert werden
sollen. Auch ein lahmlegen oder verlangsamen der Produktion kann zu monet�ren
Sch�den f�hren. Deshalb ist hier eine Sandbox ben�tzt worden in der Daten
regelm��ig aktualiesert worden sind und sie trotzdem bei Problemen resetiert
worden konnte.

\section{Semaphore / Visual Basic}
Semaphore sind eine Herausforderung da HPQC Zugriff auf die Daten mit semaphoren kann sperren und es an L�sungen gedacht werden muss die immer funktionieren.
\section{XML Parsing}

\section{Zeigen von gro�en Datenmengen auf wenig Feldern}
Die Auswahl von dem gew�nschten Element wird schwierig wenn eine Liste tausende Elemente enth�lt. Eine M�glichkeit dies zu vereinfachen ist die Liste auf mehreren Listen aufzuteielen, die dynamisch generiert werden, abh�ngig von der vorigen Auswahl.
\section{Workflow in HPQC}
Der Workflow in HPQC besteht aus gro�teil unkommentierten LOC , alles auf nur eine Seite. Dies kann dazu f�hren das leicht die �bersicht verloren geht und Fehler passieren.