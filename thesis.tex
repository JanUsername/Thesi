\documentclass[12pt,a4paper,oneside]{book}

\usepackage[ansinew]{inputenc}
\usepackage[english]{babel}
\usepackage{graphicx}
\usepackage{fourier}
\usepackage{listings}
\usepackage{hyphenat}
\usepackage{xstring}
\usepackage{forloop}
\usepackage{longtable}
\usepackage{enumitem}
\usepackage{array,booktabs}
\usepackage{float}


\lstset{basicstyle=\tt, identifierstyle=\tt, commentstyle=\tt, stringstyle=\tt, keywordstyle=\tt, language=Java, numbers=left, numberstyle=\tiny, numbersep=5pt, tabsize=2, showtabs=false, showspaces=false, showstringspaces=false, prebreak=\mbox{\tiny$\searrow$}, breakindent=0pt, breaklines, breakautoindent=true, captionpos=b}

\oddsidemargin 1,5cm
\textwidth 14,5cm 
 
\begin{document}
\frontmatter

\newsavebox\MyArrowBox%
\sbox\MyArrowBox{$\hookleftarrow$}%
\makeatletter%
\newcommand*{\BreakableChar}{%
  \leavevmode%
  \prw@zbreak%
  %\discretionary{}{}{}%
  %\discretionary{-}{}{}% If want a dash
  \discretionary{\usebox\MyArrowBox}{}{}%
  \prw@zbreak%
}%

\newcounter{index}%
\newcommand{\url}[1]{%
  \texttt{%
  \StrLen{#1 }[\stringLength]%
  \forloop[1]{index}{1}{\value{index}<\stringLength}{%
    \StrChar{#1}{\value{index}}[\currentLetter]%
    {\currentLetter\BreakableChar}%
  }%
  }%
}%

% define here the properties of the title page
\def\title{Erweiterung eines Testdokumentationstools zur besseren/einfacheren Strukturierung und Wartung dokumentierter Testf�lle}
\def\author{Jan Schmid Niederkofler}
\def\advisor{Andrea Janes}
\def\date{Juli, 2017}

\begin{titlepage}
\begin{center}\includegraphics[scale=1.4]{img/logo.pdf}\\\LARGE\vspace{80pt}BACHELOR
THESIS\vspace{50pt}\\{\huge\title}\\\vspace{50pt}\author \end{center}\Large\begin{flushright}\vspace{90pt}Supervisor: \advisor\\\date\end{flushright}
\end{titlepage}

%\input{chapters/abstract}

\tableofcontents
\listoftables
\listoffigures
\lstlistoflistings

\mainmatter

\chapter*{Abstract deutsch}

\begin{enumerate}
  \item 
 \item \textbf{Problem statement}: Den Verlauf der Testautomatisierung kann man in 2 Phasen unterteilen. Erstere
 die Dokumentation der Testf�lle und letztere das Automatisieren. W�hrend
 die bereits standartisierten Tests aus der Testfactory gewartet werden riskiert
 man durch den st�ndigen Wandel der zu Testenden Programme die Dokumentation zu
 vernachl�ssigen da diese nicht nur Zeitintensiv ist sondern auch nicht so oft
 ben�tigt wird wie ersteres, aber nicht weniger wichtig.
  

  \item \textbf{Motivation}: Die Motivation hinter diesem Projekt ist die Wartbarkeit der Dokumentation zu
  erleichtern bzw. zu erm�glichen. Dessen Funktion vielleicht nicht so viel
  Einfluss auf den Alltag hat, jedoch gro�e Hilfe bieten kann f�r den Fall das
  Fehler entdeckt werden und/oder der zust�ndige f�r den Bereich nicht Anwesend
  ist.
  

  \item \textbf{Approach}:
  Die ersten Schritte bei der Probleml�sung war das Erfassen was bereits
  vorhanden ist womit man Arbeiten kann und was selbst noch zu machen ist.
  Sobald man ein Gef�hl daf�r hat macht man sich mit der Umgebung vertraut, wie
  kann man was Steuern, welche Tools k�nnen helfen, welche Sprache ist hier
  am besten geeignet. Die Wahl fiel schnell auf VBS das es von der Applikation
  am besten unterst�tz wird. Obwohl diese in viele Hinsicht C-Sprachen
  unterlegen w�re.
  Hier ist auch n�tzlich zu Wissen wie weit man das System auslasten kann. Da
  wir hier von tausenden Elementen sprechen m�ssen wir schaugen ab wann wird
  der Ablauf durch die neue Feature langsamer oder es Limitationen vom System aus gibt.
  
  Hier hatten wir noch dazu die M�glichkeit das Tool intern oder extern zu
  erweitern, und entschieden uns auf so viel M�glich auf extern, da sich intern
  bereits �ber die Zeit sich viel Code angesammelt und die externen VBS Scripte
  einfach einzuspielen sind.
  
  Bei der Implementation ist aufzupassen welche Fehler passieren k�nnten und wie
  man mit ihnen Umgehen sollte. Da dieses Projekt gro�es Erweiterungpotential
  hat war es auch wichtig versuchen im Rahmen der Zeit zu bleiben.
  
  

  \item \textbf{Results}:   
  Das Erebnis ist eine leichtere und weniger Fehleranf�llige Eingabe bei der
  Dokumentation. Auch hat man sp�ter eine leichtere Wartung da der Pfad
  standartiesiert eingeben ist und man so Testf�lle ausmachen kann die von der
  �nderung bestimmter Masken betroffen sind. �ltere Systeme m�ssen �berarbeitet
  werden da die Daten in verschiedenen Formaten abgespeichert worden sind.

  \item \textbf{Conclusions}: What are the implications of your answer? Is it
  going to change the world (unlikely), be a significant "win", be a nice hack,
  or simply serve as a road sign indicating that this path is a waste of time
  (all of the previous results are useful). Are your results general,
  potentially generalizable, or specific to a particular case?
  
  Das erreichte Ergebnis bietet die Grundlage einer besserung Wartung. Die
  erleichtert die Eingabe bietet aber sogleich M�glichkeit ausgebaut zu werden
  um mehr Arbeiten weniger Fehleranf�llig sein zu lassen oder gar zu
  automatisieren.
  
\end{enumerate}
\chapter{Einleitung}

\section{Einleitung}
Testf�lle sind die einzelnen Tests die durchgef�hrt werden um sicherzustellen, dass Anwendungen funktionieren wie erwartet.
Die Test�lle werden erstellt von den jeweiligen Verantwortlichen erstellt und
besitzen dann einen fixen Ablauf. Mit HP Application Lifecycle Manager werden diese dokumentiert.
Durch die Erweiterung dieses verbessert sich die Wartbarkeit der dokumentierten Testf�lle. St�ndige Ver�nderungen bei den MAX Anwendungen erschweren diese. Auch verbessert wird die Eingabe der dokumentierten Testf�lle �ber vorgegebenen Dropdown Men�s.

Supply Chain Lufthansa
Lufthansa ist mit fast 126.000 Mitarbeitern eine der gr��ten Fluggesellschaften weltweit. Diese wird aufgeteilt in Passagierbef�rderung, Fracht, Technik, Catering, IT- Dienstleistungen und Service �und Finanzgesellschaft. Dieses Projekt wurde in der Lufthansa Technik realisiert. Genauer im Bereich der Supply Chain. Der Einsatzbereich der Supply Chain ist die Versorgung von Materialen f�r Kunden. Diese werden unterteilt in Material das repariert und wieder im Umlauf gebracht wird, wie ein Flugzeuggetriebe, als auch Material das einfach aufgestockt wird wie Schrauben etc.
Um den Kunden ein Interface bieten zu k�nnen �ber denen sie gew�nschte Bestellungen abgeben k�nnen, bietet Lufthansa verschiedene Software-L�sungen an. Diese unterstehen regelm��igen Updates (bis zu 7 Updates im Jahr)((source)). Um sicherzugehen das nach einem Update alles noch funktioniert wie es sollte wurden bis vor kurzen Integrationstests abgehalten, die daraus bestanden die Applikationen manuell durchzuklicken. Diese Arbeitszeit soll nun mit automatischen Tests reduziert  werden. ((insert source about Lufthansa))
\section{Testdokumentation  / Testautomatisierung}
Die Vorteile der Testautomatisierung sind in erster Linie Zeitersparnis durch die Abnahme der Testf�lle per Computer. Automatisierte Tests laufen, ohne menschliche Interaktion,  in bis zu 20 h durch((insert source, fact check with reality)). Manueller Testf�lle hingegen dauern Tage und bed�rfen Menschen die sie durchklicken. Durch die Reduzierung der menschlichen Interaktion auf ein Minimum entstehen auch weniger Probleme falls ein Experte f�r einen bestimmten Bereich nicht anwesend sein kann, sondern wird nur noch ben�tigt falls die Tests nicht erfolgreich abgeschlossen werden k�nnen. 
Ein Automatisierungszyklus beginnt mit der Auffindung aller Testf�lle. Diese werden dann dokumentiert. Der erste Schritt der Dokumentation beginnt mit einem Tool namens �SAP Workforce Performance Builder�. Dieses ist ein Werkzeug um automatisch bei jeden User Input Screenshots  zu erstellen. So kann ein Testfall in erster Instanz abgebildet sein.
Ein Schritt zur Automatisierung ist die  Dokumentation. Haupts�chlich liegt der Vorteil bei Dokumentation dabei, dass auch wenn ein Experte oder Wissenstr�ger verloren geht, nicht das Wissen selbst verloren ist.
Die Dokumentation beginnt mit �bertragung des Testfalls in das HP Tool �Application Lifecycle Manager� ((insert screenshot)). Hier werden die Schritte einzeln in eine Tabelle eingegeben. Ein einzelner Testschritt besteht aus: wo genau die Aktion stattfindet, was f�r ein Input kommt, welches Objekt an der Maske angesprochen wird. Ein Testschritt kann auch eine SQL abfrage sein oder der 
Start eines Scripts. Diese Dokumentation wird sp�ter verwendet um Testf�lle manuell durchzuspielen. 
Manuelle Tests bedarf es bei Tests die nicht zu Automatisieren sind. Parallel dazu werden manuelle Test verwendet bei fehlgeschlagenen automatisierten Tests den Fehler zu suchen.                                                                                             
 
Ist ein Testfall komplett �bertragen und kontrolliert (4 Augen Prinzip) wird er nach Budapest ((Testfactory/ Testautomatisierung))f reigegeben, wo mit Hilfe von Visual Basic in Script erstellt wird, der den Testfall nachspielt. Dieses Script ist unabh�ngig von der Dokumentation im HPQC, auch wenn dieses darauf basiert. �nderungen an der Maske werden nur in den automatisierten Testf�llen eingespielt. 

Object Action genauer beschreiben + Beschreibungssprache und Formale Sprache (aus dem Wiki) + outsourcing probleme

\section{Ausgangssituation}
Die �bertragung der Testf�lle in HP Quality Center funktioniert, ist durch Textfeld basierten Input aber Fehleranf�llig. Noch dazu ist der Aufwand bei �nderungen zu hoch, so dass diese bei der gro�en Anzahl an Testf�llen die Dokumentation Anpassung  ganz wegbleibt. Die betrifft jetzt nicht automatisierte Tests. Hier das Visual Basic Script angepasst.
((Es funktioniert aber k�nnte besser sein, bei Pflege))
\section{Schwer bis unm�glich manuelle Testf�lle zu warten}
((Anders Beginnen, kein geeigneter erster Unterpunkt, braucht mehr roten Faden))
Ein entscheidender Punkt in der Dokumentation ist die Angabe wo ein bestimmtes Element sich befindet. Dies wird mit Object Path und Object Name angegeben. Bei der �bertragung in das HP tool wird dies manuell eingepflegt. �ndert sich die Maske, durch die regelm��igen Updates, wird diese Angabe falsch. Dies richtig zu Pflegen beinhaltet alle Testf�lle durchzugehen und wo auf die ge�nderte Oberfl�che zugegriffen wird dementsprechend zu �ndern. Folglich sind diese Tests nicht skalierbar und bereits bei wenigen Testf�llen ist mit �nderungen gro�er Aufwand verbunden.

\chapter{Grundlagen und Stand der Technik}
\section{Hp Tool}
Wenn du ein �HP Tool� erw�hnst, musst du dieses Tool kurz erkl�ren. Das w�rde in �Grundlagen und Stand der Technik� gut hineinpassen.
Das HP Tool Application Life Cycle Manager oder kurz HP ALM ist eine Sammlung von Software Applikationen die den Lebenszyklus von Software unterst�tzt.
 In diesen Fall die Erstellung von automatisierten Testf�llen. HP. 
 Spezifisch wird hier HP Quality Center verwendet das eine einzige Plattform f�r
 Test- und Software Zyklus Management bereitstellt. Die Aufgabe der Software
 Factory ist nicht die genutze Software nachzubasteln sondern eigentst�ndige
 Programme schreiben die auf einer existierend Testumgebung die gew�nschten
 Abl�ufe rekonstruiert. 



 
 
 
 TODO:
 -> Outsourcing theorie, zumindest a bissl
  4) Grundlagen -> Formale Sprache, Object Path ersatz, find that document->gefindet

  [insert source and more explanation about hp here + Cycles?]
\section{UFT}
\chapter{Herausforderungen}

[In der Zusammenfassun so lange nicht aktive Probleme. Dargestellt wie in Zukunft ausbauf�hig und k�nnte und den Umst�nden Probleme machen. Alles komplett abschlie�en, keine Offenen Enden]
\section{Rechte in einen Gro�konzern} 
W�hrend der Entwicklung des tools wird das Hp Quality Center Programm noch
weiterhin verwendet. In diesen sind Daten gespeichert die nicht ver�ndert werden
sollen. Auch ein lahmlegen oder verlangsamen der Produktion kann zu monet�ren
Sch�den f�hren. Deshalb ist hier eine Sandbox ben�tzt worden in der Daten
regelm��ig aktualiesert worden sind und sie trotzdem bei Problemen resetiert
worden konnte.

\section{Semaphore / Visual Basic}
Visual Basic ist konzeptiert als Script Sprache die Aufgaben automatisiert
ausf�hren kann, oder f�r Zeitersparnis sorgen kann. Sie ist von HP ausgew�hlt
worden um als Schnittstelle f�r die Life Cycle Applikation zu fungieren bzw. sie
auf Wunsch ma�zuschneidern. Daf�r bietet Hp eine Reihe von Bibliotheken mit
Zugang auf allen m�glichichen Bereichen, sei es von extern �ber Scripts, sei es
es intern �bern den Hauseigenen Workflow.

Mit dies erweist sich VBS als eindeutig geigneste Wahl f�r dieses Projekt. Nun
hat eine Scritpsprache gegen�ber anderen Programmiersprachen wie Java einige
Nachteile. Einmal ist der einzige Datentyp Variant, dieser kann zu Problemen
f�hren wenn sich HPQC bestimmte Datentypen erwartet aber VBS sie nicht richtig
interpretiert. Dies muss dann richtig konvertiert werden oder es kann zu Fehlern
kommen. 

Ein weiterer m�glicher Konfliktpunkt k�nnte der Mangel an Abstrakten Datentypen
sein. Da wir mit gr��eren Mengen an Daten mit vielen einzelnen und untereinander
verkn�pften Elementen zu arbeiten haben w�ren diese eine nat�rliche Wahl. Hier
m�ssen wir Alternativen suchen die z.B. mit String manipultation funktioniert.

\section{XML}
Eine weitere Herausforderung bringen die XML Dateien. Mit VBS verlieren wir
die ``Tiefe'' der Elemente. Da einmal eingelesen VBS alles Zeile f�r Zeile
behandelt und keinen unterschied macht ob untereinander Zeilen verschachtelt
sind. Sie macht die XML Datei eindimensional. Hier w�ren ADT n�tzlich beim
einlesen. Gel�st wurde es mit der Namen der Listen bzw. der Elemente auf denen
die vorhergehende Ebene verweist.

Des wurde das Format wie es jetzt standartisiert ist erst im Laufe der Zeit
entwickelt und �ltere XML Dateien sind grunds�tzlich sehr verschieden aufgebaut.
Um dies entgegen zu wirken m�ssen diese auf dem neusten Stand gebracht werden.

\section{Zeigen von gro�en Datenmengen auf wenig Feldern}
Schwierigkeiten kann auch eine verst�ndlich Darstellung der eingespielten Daten
bringen. Diese gehen in die Tausende Die Auswahlm�glichkeiten sind auf
Eingabefelder und Comboboxen beschr�nkt. So kann es schnell passieren das einem
die �bersicht verloren geht und die geplante vereinfachte Eingabe das Gegenteil
erreicht. Eine M�glichkeit ist mehrere Comboboxen, an Stelle wie vorher einem
Textfeld, die sich nach unten skalieren und die Auswahl mit einent Top-down
approach vergleichbar ist. Es wird von der Applikation �ber Panel zu einzelnen
Objekten immer die n�chste Ebene �ber ausgegraut und mit der vorhergehenden
Auswahl schr�nkt sich die Auswahl genug ein um �bersichtlich zu bleiben. Die
n�chsten Auswahlm�glichkeiten werden dynamisch generiert im Workflow.

\section{Workflow in HPQC}
�ber den Workflow in HPQC lassen sich Elemente ansteuern und ver�ndern. Dieser
ist mit der Zeit gewachsen, da das Programm auf eigene W�nsche ma�geschneidert
worden ist. So bestand das Risiko den noch weiter aufzublasen was Wartung und
Entwicklung erschweren k�nnten. Deshalb wurde beschlossen Soviel wie m�glich mit
externen Scripten zu machen. 

Diese sind einfach einzuspiele und aufzubewaren. Aufgepasst werden muss das
Arbeiten nicht doppelt verrichtet werden, einmal intern und einmal extern, das
wiederum dazu f�hren w�rden die �bersicht zu verlieren. 

\input{chapters/L�sungen}
\chapter*{Problemstelliung}
3. Problemstellung
3.2 Warum HP Quality Center
Durch eine einzige Plattform die dieses Tool f�r Lebenszyklen und Test Management bereitstellt macht es f�r die Testautomatisierung geeignet. Die Testdokumentation und dessen Automatisierung folgt gewissen Zyklen [explaining here or somewhere else, or at all? ] die sich in HP QC unter �Status� wiedergeben l�sst. Das Projekt ist nicht zentral und wird teilweise outsourcend und bei Setzung dieses Status wei� wann welche Personen jetzt f�r den n�chsten Schritt verantwortlich sind. Diese werden benachrichtigt sobald dieser in den eigenen Aufgabenbereich f�llt.
3.3 Requierments
Im Rahmen dieser Arbeit wird �ber UFT eine XML aus den bereits automatisierten Testf�llen alle bekannten Objekten ausgelesen und �ber in ein Object Repostory eingespielt. Auf diesen wird �ber
den ALM Explorer mit Dropdown Men�s dem Endnutzer Zugriff gegeben. Der gesamte Pfad zu einem Objekt hin muss �ber dieselbe HP QC GUI erkenntlich sein.
Es muss eine M�glichkeit bestehen Objekte die neu hinzugekommen sind trotzdem einzuspielen. Am besten ohne externe Anbindungen und direkt �ber den Input an der grafischen Oberfl�che.
Im Besonderen darf das Projekt HP QC nicht deutlich langsamer machen. Wird das Tool bis zur Unbrauchbarkeit langsamer �berwiegen die Nachteile den Vorteilen und das Projekt wird �berfl�ssig.
[Noch nicht sicher wo das hingeh�rt]
\chapter{Implementation}
\section{OTA / tdconnect, Visual Basic Script}
Die Open Test Architecture API (OTA) erlaubt es mit tdconnect Zugriff auf HP
Application Lifecycle Manager �ber externe Scripte zu haben. Die Schnittstelle
wird mit Visual Basic Script angesprochen. F�r diesen Zweck wurde Visual Basic
Script C(sharp) bevorzugt da diese Sprache f�r HPQC am geeignetsten ist. VBS ist
gut dokumentiert und verf�gt �ber eine viel gr��ere Bibliothek was die Steuerung des
Tools anbelangt. Der Workflow von HPQC ist auch mit VBS beschrieben, was eine
Andockung mit der selbigen vereinfacht. C(sharp) scheint zu diesen Zeitpunkt
noch nicht f�r die Software ausgearbeitet genug zu sein. Das kann sich in Zukunft vielleicht �ndern.
\section{ Listen adden + Elemente}
\begin{lstlisting}
' Add list under costumization -> Project Lists;
Sub AddList(tdConnection)
MsgBox("adding lists like a champ")
' custom is the connection with the elements, oList are the lists access
Dim custom
Set custom = tdConnection.Customization
Dim oLists
Set oLists = custom.Lists
Dim nameList
Dim counter
counter = 0
While counter < 10
nameList = "LHT_Test_Nr " & counter
oLists.AddList (nameList)
'AddList RemoveList; Adding or removing List to the queue on the list object
Call AddItemToList(tdConnection, nameList) �'adding an item to thelist, does not work if node is in queue and not commisted
counter = counter + 1
Wend
custom.Commit
End Sub
[Code beschreiben]
\end{lstlisting}
\section{Aus XML auslesen}
\section{Ergebnis}
\chapter{Future}

\section{Zuk�nftiger Einsatz und Ausbau}
\section{Allen Testf�llen}
Im Moment k�nnen nicht alle Testf�lle standartisiert werden da �ltere XML
Dateien andere Formate bzw. der Standard sich erst mit der Zeit so wie
er ist entwickelt hat. Diese m�ssen erst angepasst werden um �ltere
bereits eingespielte Tests sp�ter leichter zu warten sind. 
[Anstatt
dokumentierte Testf�lle -> Dokumentation �Synonyme� -> Standardisieren, einheitliche Beschreibung der Testf�lle, Formalisieren-> sollen alle gleich Aussehen]

\section{ Manuelle Eingabe aus der End-User Maske}
Zuk�nftige Ver�nderung an der Graphischen Oberfl�che wir nicht automatisch bei
der Testautomatisierung mitver�ndert. Bei dem �bertragen eines Testfalles wird
anstelle des Pfades ein Feld Sonstiges ausgef�llt, das sp�ter von der
Testfactory eingespielt wird. Eine m�glich Feature besteht darin End-user direkt
�ber die Eingabemaske die Dateien im Hintergrund zu ver�ndern. Trotzdem wird
eine R�ckbest�tigung der Testautomatiser sicherer sein.

\section{Direkt Automatisiert}
Im HP ALM sind jetzt Listen aller M�glichen Objekte und den Customisierbaren
Listen gepseichert, vorher waren diese nur extern �ber XML zug�nglich. Die
bietet einen m�glich Ausbau f�r die Testautomation um diese teil oder volls vom
HPQC �bernehmen zu lassen.

\input{chapters/Zusammenfassung-Fazit}


\appendix

\bibliographystyle{plain}                                 
\bibliography{bibliography}

\end{document}